\documentclass[conference]{IEEEtran}
\IEEEoverridecommandlockouts
% The preceding line is only needed to identify funding in the first footnote. If that is unneeded, please comment it out.
\usepackage{polski}
\usepackage[utf8]{inputenc}
\usepackage{todonotes}

\usepackage{cite}
\usepackage{amsmath,amssymb,amsfonts}
\usepackage{algorithmic}
\usepackage{graphicx}
\usepackage{textcomp}
\usepackage{xcolor}
\def\BibTeX{{\rm B\kern-.05em{\sc i\kern-.025em b}\kern-.08em
    T\kern-.1667em\lower.7ex\hbox{E}\kern-.125emX}}
\begin{document}

\title{Skaner 3D wykorzystujący linię laserową\\}


\author{\IEEEauthorblockN{1\textsuperscript{st} Jarosław Mądry}
\IEEEauthorblockA{\textit{IARII} \\
\textit{Politechnika Poznańska}\\
Poznań, Polska \\
jaroslaw.madry@student.put.poznan.pl}
\and
\IEEEauthorblockN{2\textsuperscript{nd} Janusz Sendyk}
\IEEEauthorblockA{\textit{IARII} \\
\textit{Politechnika Poznańska}\\
Poznań, Polska \\
janusz.sendyk@put.poznan.pl}
\and
\IEEEauthorblockN{3\textsuperscript{rd} Given Name Surname}
\IEEEauthorblockA{\textit{dept. name of organization (of Aff.)} \\
\textit{name of organization (of Aff.)}\\
City, Country \\
email address}

}

\maketitle

\begin{abstract}
Dokument opisuje projekt optycznego skanera 3D wykorzystującego oświetlenie strukturalne w postaci linii laserowej. Skaner ten przystosowany został do umieszczenia na ramieniu robota manipulacyjnego UR-5.

\end{abstract}

\begin{IEEEkeywords}
skan, 3d, laser, laser plane, structured light
\end{IEEEkeywords}

\section{Wstęp}

Podstawa działania programu opiera się o równania pozwalające wyznaczyć położenie punktu w przestrzeni leżącego na znanej płaszczyźnie na podstawie położenia na obrazie z kamery.
$\underline{X}={P_k^+}\underline{x}+\lambda 
\begin{bmatrix}
    0\\
    0\\
	0\\
    1
\end{bmatrix}$\\
$\underline{p^T}({P_k^+}\underline{x}+\lambda*\underline{C})=0$
\\
przy czym:
\\
$\underline{X}$ - współrzędne punktu w przestrzeni\\
$P_k=
\begin{bmatrix}
    fx&0&cx\\
    0&fy&cy\\
	0&0&1\\
\end{bmatrix}\cdot
\begin{bmatrix}
    1&0&cx&0\\
    0&1&cy&0\\
	0&0&1&0\\
\end{bmatrix}$ - macierz parametrów kamery\\
${P_k^+}$ - pseudo odwrotna macierz kamery\\
$\underline{x}$ - współrzędne punktu na obrazie\\
$\underline{C}=
\begin{bmatrix}
    0\\
    0\\
	0\\
    1
\end{bmatrix}$ - współrzędne kamery\\
$\underline{p}^t = 
\begin{bmatrix}
    n_x\\
    n_y\\
	n_z\\
    -d
\end{bmatrix}$ - wektor płaszczyzny

Na podstawie 2 równania wyznaczana jest $\lambda$ a następnie wykorzystując pierwsze równanie położenie punktu w przestrzeni 3D.

\section{Mechanika}

\subsection{Maintaining the Integrity of the Specifications}

The IEEEtran class file is used to format your paper and style the text. All margins, 

\section{Elektronika}
Projekt wymagał niedużej ilości operacji związanych z elektroniką.

Jako zasilanie kamery postanowiono wykorzystać istniejący kabel Ethernet oraz fakt obsługiwania przez zastosowane urządzenie standardu PoE. Rozwiązanie to pozwala na uniknięcie dodatkowych przewodów zasilających do kamery.

Do zasilenia lasera wykonano odłączany przewód z wyjściami bananowymi do zasilenia z zasilacza laboratoryjnego. Napięcia zasilania lasera wynosi 5V.

\section{Program - Visual}

Program został napisany w środowisku VisualStudio 2017

Program wykorzystuje biblioteki:
\begin{itemize}
\item OpenCV v3.4.0
\item Eigen v3.3.3
\item Pylon v5.0
\end{itemize}

Do działania algorytmu niezbędne są informacje z kalibracji kamery zawierające macierz kamery oraz macierz deformacji, oraz równanie płaszczyzny lasera.

\subsection{Kalibracja kamery}

Po uruchomieniu program sprawdza czy istnieje plik z danymi o kalibracji kamery. Jeżeli nie to uruchamia procedurę kalibracji. 

Kalibracja wymaga szachownicy kalibracyjnej, której wymiary są zdefiniowane w zmiennych $"checkerboard\_size\_x"$ oraz $"checkerboard\_size\_x"$. W trakcie kalibracji należy pobrać odpowiednią liczbę obrazów (zdefiniowaną przez $"camera\_calib\_img\_count"$) zawierających tablicę kalibracyjną w różnych obszarach widzenia kamery. Pobieranie następuje poprzez wciśnięcie dowolnego klawisza. Po pobraniu obrazów następuje proces kalibracji z wykorzystanie funkcji dostępnej w OpenCV $"calibrateCamera"$. W zależności od ilości pobranych klatek proces ten może trwać od kilku minut wzwyż. Po wyznaczeniu parametrów kamery następuje zapis do pliku $"cam_param.yml"$.

\subsection{Wyznaczenie płaszczyzny lasera}

W przypadku gdy wczytane zostały parametry kamery sprawdzane jest czy istnieje plik zawierający informacje o położeniu płaszczyzny lasera $"PlaneEq.yml"$.

Kalibracja płaszczyzny lasera wymaga szachownicy kalibracyjnej znajdującej się na płaszczyźnie ale nie zajmującej całego obszaru widzianego przez kamerę. Wynika to z tego, iż gdy szachownica oświetlona jest laserem, to nie jest ona rozpoznawana przez algorytm.
Po wciśnięciu klawisza 'p' następuje przechwycenie klatki. Na obrazie następuje wyszukanie szachownicy poprzez funkcję $"calculateCheckerboardVector"$, która to zwraca równanie płaszczyzny tablicy kalibracyjnej względem kamery. Oraz wyszukanie linii lasera $"find\_laser"$ w formie wektora punktów 2D. Następnie wykorzystując równanie płaszczyzny i punkty 2D obliczane jest położenie tych punktów w przestrzeni wykorzystując $"projectImagePointsOntoPlane"$ (funkcja działa na zasadzie opisanej we wstępie). Gdy zebrana jest chmura punktów, można dokonać dopasowania płaszczyzny funkcją $"best_plane_from_points"$. Gdy zebrana jest wystarczająca ilość punktów i dokładność płaszczyzny powinna być zadowalająca można zapisać plik z równaniem płaszczyzny lasera poprzez przytrzymanie klawisza 's'.
Testowano po 4 krotnym zebraniu punktów i równanie płaszczyzny zostało wyznaczone względnie poprawnie (nie badano dokładności).

\subsection{Skanowanie}

W przypadku gdy istnieją pliki kalibracyjne kamery oraz płaszczyzny program przechodzi do skanowania. Po wciśnięciu klawisza 'n' pobierany jest obraz, wyszukiwana jest linia lasera, obliczane jest położenie punktów lasera w przestrzeni oraz są one dodawane do chmury punktów. Tak uzyskaną chmurę można zapisać przy pomocy klawisza 's'.

\subsection{projectImagePointsOntoPlane}

Należy zaznaczyć, iż powyższa funkcja o którą opiera się cały program została dostarczona przez Jan Wietrzykowski.

\begin{thebibliography}{00}
\bibitem{b1} G. Eason, B. Noble, and I. N. Sneddon, ``On certain integrals of Lipschitz-Hankel type involving products of Bessel functions,'' Phil. Trans. Roy. Soc. London, vol. A247, pp. 529--551, April 1955.
\bibitem{b2} J. Clerk Maxwell, A Treatise on Electricity and Magnetism, 3rd ed., vol. 2. Oxford: Clarendon, 1892, pp.68--73.
\bibitem{b3} I. S. Jacobs and C. P. Bean, ``Fine particles, thin films and exchange anisotropy,'' in Magnetism, vol. III, G. T. Rado and H. Suhl, Eds. New York: Academic, 1963, pp. 271--350.
\bibitem{b4} K. Elissa, ``Title of paper if known,'' unpublished.
\bibitem{b5} R. Nicole, ``Title of paper with only first word capitalized,'' J. Name Stand. Abbrev., in press.
\bibitem{b6} Y. Yorozu, M. Hirano, K. Oka, and Y. Tagawa, ``Electron spectroscopy studies on magneto-optical media and plastic substrate interface,'' IEEE Transl. J. Magn. Japan, vol. 2, pp. 740--741, August 1987 [Digests 9th Annual Conf. Magnetics Japan, p. 301, 1982].
\bibitem{b7} M. Young, The Technical Writer's Handbook. Mill Valley, CA: University Science, 1989.
\end{thebibliography}

\end{document}
