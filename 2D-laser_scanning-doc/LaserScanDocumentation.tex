\documentclass[conference]{IEEEtran}
\IEEEoverridecommandlockouts
% The preceding line is only needed to identify funding in the first footnote. If that is unneeded, please comment it out.
\usepackage{polski}
\usepackage[utf8]{inputenc}
\usepackage{todonotes}

\usepackage{cite}
\usepackage{amsmath,amssymb,amsfonts}
\usepackage{algorithmic}
\usepackage{graphicx}
\usepackage{textcomp}
\usepackage{xcolor}
\def\BibTeX{{\rm B\kern-.05em{\sc i\kern-.025em b}\kern-.08em
    T\kern-.1667em\lower.7ex\hbox{E}\kern-.125emX}}
\begin{document}

\title{Skaner 3D wykorzystujący linię laserową\\}


\author{\IEEEauthorblockN{1\textsuperscript{st} Jarosław Mądry}
\IEEEauthorblockA{\textit{IARII} \\
\textit{Politechnika Poznańska}\\
Poznań, Polska \\
jaroslaw.madry@student.put.poznan.pl}
\and
\IEEEauthorblockN{2\textsuperscript{nd} Janusz Sendyk}
\IEEEauthorblockA{\textit{IARII} \\
\textit{Politechnika Poznańska}\\
Poznań, Polska \\
janusz.sendyk@put.poznan.pl}
\and
\IEEEauthorblockN{3\textsuperscript{rd} Given Name Surname}
\IEEEauthorblockA{\textit{dept. name of organization (of Aff.)} \\
\textit{name of organization (of Aff.)}\\
City, Country \\
email address}

}

\maketitle

\begin{abstract}
Dokument opisuje projekt optycznego skanera 3D wykorzystującego oświetlenie strukturalne w postaci linii laserowej. Skaner ten przystosowany został do umieszczenia na ramieniu robota manipulacyjnego UR-5.

\end{abstract}

\begin{IEEEkeywords}
skan, 3d, laser, laser plane, structured light
\end{IEEEkeywords}

\section{Wstęp}

\todo{Opisać zasadę działania}

\section{Mechanika}

\subsection{Maintaining the Integrity of the Specifications}

The IEEEtran class file is used to format your paper and style the text. All margins, 

\section{Elektronika}
Projekt wymagał niedużej ilości operacji związanych z elektroniką.

Jako zasilanie kamery postanowiono wykorzystać istniejący kabel Ethernet oraz fakt obsługiwania przez zastosowane urządzenie standardu PoE. Rozwiązanie to pozwala na uniknięcie dodatkowych przewodów zasilających do kamery.

Do zasilenia lasera wykonano odłączany przewód z wyjściami bananowymi do zasilenia z zasilacza laboratoryjnego. Napięcia zasilania lasera wynosi 5V.

\section{Program - Visual}

Program został napisany w środowisku VisualStudio 2017

Program wykorzystuje biblioteki:
\begin{itemize}
\item OpenCV v3.4.0
\item Eigen v3.3.3
\item Pylon v5.0
\end{itemize}

Do działania algorytmu niezbędne są informacje z kalibracji kamery zawierające macierz kamery oraz macierz deformacji, oraz równanie płaszczyzny lasera.

\subsection{Kalibracja kamery}

Po uruchomieniu program sprawdza czy istnieje plik z danymi o kalibracji kamery. Jeżeli nie to uruchamia procedurę kalibracji. 

Kalibracja wymaga szachownicy kalibracyjnej, której wymiary są zdefiniowane w zmiennych "checkerboard_size_x" oraz "checkerboard_size_x". W trakcie kalibracji należy pobrać odpowiednią liczbę obrazów (zdefiniowaną przez "camera_calib_img_count") zawierających tablicę kalibracyjną w różnych obszarach widzenia kamery. Pobieranie następuje poprzez wciśnięcie dowolnego klawisza. Po pobraniu obrazów następuje proces kalibracji z wykorzystanie funkcji dostępnej w OpenCV "calibrateCamera". W zależności od ilości pobranych klatek proces ten może trwać od kilku minut wzwyż. Po wyznaczeniu parametrów kamery następuje zapis do pliku "cam_param.yml".

\subsection{Wyznaczenie płaszczyzny lasera}



\subsection{Figures and Tables}
\paragraph{Positioning Figures and Tables} Place figures and tables at the top and 
bottom of columns. Avoid placing them in the middle of columns. Large 
figures and tables may span across both columns. Figure captions should be 
below the figures; table heads should appear above the tables. Insert 
figures and tables after they are cited in the text. Use the abbreviation 
``Fig.~\ref{fig}'', even at the beginning of a sentence.

\begin{table}[htbp]
\caption{Table Type Styles}
\begin{center}
\begin{tabular}{|c|c|c|c|}
\hline
\textbf{Table}&\multicolumn{3}{|c|}{\textbf{Table Column Head}} \\
\cline{2-4} 
\textbf{Head} & \textbf{\textit{Table column subhead}}& \textbf{\textit{Subhead}}& \textbf{\textit{Subhead}} \\
\hline
copy& More table copy$^{\mathrm{a}}$& &  \\
\hline
\multicolumn{4}{l}{$^{\mathrm{a}}$Sample of a Table footnote.}
\end{tabular}
\label{tab1}
\end{center}
\end{table}

\begin{figure}[htbp]
%\centerline{\includegraphics{fig1.png}}
\caption{Example of a figure caption.}
\label{fig}
\end{figure}

Figure Labels: Use 8 point Times New Roman for Figure labels. Use words 
rather than symbols or abbreviations when writing Figure axis labels to 
avoid confusing the reader. As an example, write the quantity 
``Magnetization'', or ``Magnetization, M'', not just ``M''. If including 
units in the label, present them within parentheses. Do not label axes only 
with units. In the example, write ``Magnetization (A/m)'' or ``Magnetization 
\{A[m(1)]\}'', not just ``A/m''. Do not label axes with a ratio of 
quantities and units. For example, write ``Temperature (K)'', not 
``Temperature/K''.


\section*{References}

Please number citations consecutively within brackets \cite{b1}. The 
sentence punctuation follows the bracket \cite{b2}. Refer simply to the reference 
number, as in \cite{b3}---do not use ``Ref. \cite{b3}'' or ``reference \cite{b3}'' except at 
the beginning of a sentence: ``Reference \cite{b3} was the first $\ldots$''

Number footnotes separately in superscripts. Place the actual footnote at 
the bottom of the column in which it was cited. Do not put footnotes in the 
abstract or reference list. Use letters for table footnotes.

Unless there are six authors or more give all authors' names; do not use 
``et al.''. Papers that have not been published, even if they have been 
submitted for publication, should be cited as ``unpublished'' \cite{b4}. Papers 
that have been accepted for publication should be cited as ``in press'' \cite{b5}. 
Capitalize only the first word in a paper title, except for proper nouns and 
element symbols.

For papers published in translation journals, please give the English 
citation first, followed by the original foreign-language citation \cite{b6}.

\begin{thebibliography}{00}
\bibitem{b1} G. Eason, B. Noble, and I. N. Sneddon, ``On certain integrals of Lipschitz-Hankel type involving products of Bessel functions,'' Phil. Trans. Roy. Soc. London, vol. A247, pp. 529--551, April 1955.
\bibitem{b2} J. Clerk Maxwell, A Treatise on Electricity and Magnetism, 3rd ed., vol. 2. Oxford: Clarendon, 1892, pp.68--73.
\bibitem{b3} I. S. Jacobs and C. P. Bean, ``Fine particles, thin films and exchange anisotropy,'' in Magnetism, vol. III, G. T. Rado and H. Suhl, Eds. New York: Academic, 1963, pp. 271--350.
\bibitem{b4} K. Elissa, ``Title of paper if known,'' unpublished.
\bibitem{b5} R. Nicole, ``Title of paper with only first word capitalized,'' J. Name Stand. Abbrev., in press.
\bibitem{b6} Y. Yorozu, M. Hirano, K. Oka, and Y. Tagawa, ``Electron spectroscopy studies on magneto-optical media and plastic substrate interface,'' IEEE Transl. J. Magn. Japan, vol. 2, pp. 740--741, August 1987 [Digests 9th Annual Conf. Magnetics Japan, p. 301, 1982].
\bibitem{b7} M. Young, The Technical Writer's Handbook. Mill Valley, CA: University Science, 1989.
\end{thebibliography}

\end{document}
